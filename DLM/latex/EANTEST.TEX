% Quelle : CTAN://macros/generic/ean

% Wenn ein \ISBN vor einem \EAN steht,
% dann wird zus�tzlich die ISBN-Nummer �ber dem Barcode gedruckt. Aber
% erst, wenn auch tats�chlich ein \EAN benutzt wird. Das
% \ISBN alleine druckt nichts. Es ist also eher ein Parameter f�r den
% folgenden \EAN

% Was tun die ``-'' im Barcode?

% *Muss* man das \X definieren? Ja. Das \X definiert die Breite des
% Strichmoduls.
%
%                   \X=n         \font... scaled n
% �������������������������������������
% Standard = 100% � \X=0.330mm � 1000
%                 �            �
%  oder, wenn ich eine Variable <Scale> habe, die zwischen 80 und 200
%  sein kann:
% \X=0.33*<Scale>/100
% \font\ocrb=ocrb9 scaled <Scale>*10

\documentclass[ngerman]{article}
\begin{document}

\input ean13

% \X=0.396mm               % the module size is 120% of standard

%% font used to print the number below bars
\font\ocrb=ocrb9 scaled 400

%% re-load ISBN font to the new size
% \font\ocrbsmall=ocrb7 scaled 400


%% bar correction (?)
%% \bcorr=0.015mm 

\ISBN 80-901950-0-4
%% the bigskip does not insert any space 
\bigskip 

% \X=0.132mm               % the module size is 40% of standard
\barheight=7mm                  % this must be after \ISBN call
\EAN 978-80-901950-0-4          % produces barcode with ISBN

                                % \ISBN 80-901950-0-8

%\EAN 978-80-901950-0-4 % Typesetting System TeX

\vskip1cm

\input ean8

\EAN 8591-2342
\EAN 8-5-912342

\end{document}

